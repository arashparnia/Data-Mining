\documentclass{article}
\usepackage[utf8]{inputenc}
\usepackage{lipsum}
\usepackage{titlesec}

\titlespacing\section{0pt}{12pt plus 4pt minus 2pt}{0pt plus 2pt minus 2pt}
\titlespacing\subsection{0pt}{12pt plus 4pt minus 2pt}{0pt plus 2pt minus 2pt}
\titlespacing\subsubsection{0pt}{12pt plus 4pt minus 2pt}{0pt plus 2pt minus 2pt}

\begin{document}


\section{Titanic}
\subsection{Introduction}

In this section the data of the passengers of the titanic are analysed. Moreover based on the training data 
set, data mining techniques such as association rules, logistic regression and random forests are used to 
predict whether passengers from the testing data set, did or did not survive. The code used is called 
Titanic019 and is written in R. Most plot's are not attached to the report, some are given in the Appendix.

\subsection{Data set Exploration}
The data set used was collected from Kaggle. The data includes a training and test set. The data consists of 12
variables and is already relative clean. There are missing observations in the Age and Cabin variables. The 
Name and Cabin variable include a title and a letter respectively. All rows are unique, even when the training 
and test data sets are combined.
The variables from the Raw data set are Integers, Numerical and Factors, while some variables should be 
transformed to serve their purpose well.

\subsubsection{Distributions and Correlations}
Distributions and pie's of the data are given in the Appendix. There are mostly men, many of age 40. There were more people in first class than in second class.  Woman, children, higher economic classes, certain cabin's and titles had a better chance to survive. Correlations with the variable Survive are especially 'confident' when variables are combined. See for an example figure XXX and XXX. An correlogram is also given in the Appendix.

\subsubsection{Missing observations}

The missing observations in the Cabin data are replaced with the value 'Unknown' and treated as a group on its 
own. The single missing observation in the test data set for the variable 'Fare' is replaced with the median 
Fare. 

For the Age variable, the missing observations are predicted using the ANOVA with independent variables. 
While the variable do not have the same variance or follow a normal distribution, the predictions made were 
still reasonable and efficient. 
\begin{equation}
    Age_i= Pclass_i + Sex_i + SibSp_i + Parch_i + Fare_i + Embarked_i + title_i + \epsilon 
\end{equation}


\subsubsection{Data Transformations and Selection}

Only the title and letter in the Name and Cabin variable are used. For certain purposes the data is factorized.
For efficiency in certain data mining techniques, data is clustered (e.g. the Age variable is split in 5 
groups). Only the title and letter in the 'Name' and 'Cabin' variable are taken, the variables themselves are 
deleted. The variables 'PassengerId' and 'Ticket' are also removed while no use full purpose for them could be 
found. 

The variable 'FamilySize' is created from number of Siblings and Parents.
Some variables still could have the same meaning as other variables (e.g. title= Mr. and Sex=Male).




\subsection{Learning Algorithm's}

\subsubsection{Association Rules}

\subsubsection{Logistic Regression}

\subsubsection{Random Forest}








\end{document}



